\documentclass[12pt]{article}

\usepackage{sbc-template}
\usepackage{graphicx,url}
\usepackage[utf8]{inputenc}
\usepackage[brazil]{babel}
%\usepackage[latin1]{inputenc}  

     
\sloppy

\title{Implementação 2 - TAD's}

\author{Laura Costa\inst{1}, Lívia Xavier\inst{1}, Luca Gonzaga\inst{1}, Sophia Carrazza \inst{1} 
}


\address{Instituto de Ciências Exatas e Informática -- Pontifícia Universidade Católica de Minas Gerais
  \email{}
}

\begin{document} 

\maketitle
\section{Informações Gerais}

O presente trabalho busca descrever a implementação de um gerador de todos os subgrafos de um grafo completo com \textbf{n} vértices, em que \textbf{n} é informado pelo usuário. Além disto, informando o número de subgrafos gerados.  utilizando a linguagem C++. 
\section{Estruturas Básicas}
\begin{enumerate}
    \item Criamos uma função chamada geradorDeSubgrafos que recebe um inteiro n, informando o número de vértices de um determinado grafo.
    \item O número total de arestas em um grafo completo não direcionado. utilizamos em código a variável nArestas para represntar isso: 
    \begin{enumerate}
        \item nArestas =n * (n - 1) / 2;
    \end{enumerate}
    \item Também utilizamos um contador dos subgrafos para entregar o resultado da tarefa proposta.
    \begin{enumerate}
        \item nSubgrafos = 0
    \end{enumerate}
\item  Váriavel booleana subgrafoValido apenas para retornar se o subgrafo verificado existe ou não
\end{enumerate}


\section{O código}
\begin{enumerate}
    \item Criamos um vetor que armazena todas as arestas do grafo. 
    \item  primeiro verificamos todas as combinações possíveis de vértices de um grafo completo a partir da fórmula: \textit{( \( 2^n - 1 \))}
ele faz a combinação entre todos os vértices utilizando o número inserido pelo usuário e depois faz o mesmo com as arestas a partir da fórmula: \( 2^{nArestas} \). 

    \item Certo, agora verificamos e selecionamos todos os "vértices ativos", ou seja, o código verifica quais vértices estão presentes no subgrafo atual. Isso é feito com uma operação de bitwise, onde cada bit de combinacaoVertices representa um vértice.
    \item Após isso é feito uma conferência de todas as arestas para verificar se são válidas. Isso porque, se a aresta foi selecionada mas algum dos seus vértices nao ela é desconsiderada. Esse processo é feito a partir da verificação de ambas as extremidades da aresta, caso alguma delas não esteja no subgrafo atual retorna \textit{subgrafoValido} = false
    \item E para finalizar se o subgrafo for válido ele é printado e contabilizado na váriavel \textit{nSubgrafos}.
    \item Após toda a execução do programa o número total de subgrafos existente a partir de um grafo completo é printado na tela.
\end{enumerate}


\end{document}
